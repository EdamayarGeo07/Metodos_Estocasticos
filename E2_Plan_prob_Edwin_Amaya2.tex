% Generated by GrindEQ Word-to-LaTeX 
\documentclass{article} % use \documentstyle for old LaTeX compilers

\usepackage[utf8]{inputenc} % 'cp1252'-Western, 'cp1251'-Cyrillic, etc.
\usepackage[english]{babel} % 'french', 'german', 'spanish', 'danish', etc.
\usepackage{amsmath}
\usepackage{amssymb}
\usepackage{txfonts}
\usepackage{mathdots}
\usepackage[classicReIm]{kpfonts}
\usepackage{graphicx}
\usepackage[margin=1in]{geometry}


% You can include more LaTeX packages here 

\begin{document}

%\selectlanguage{english} % remove comment delimiter ('%') and select language if required

\title{\textbf{CARACTERIZACI\'{O}N ESTAD\'{I}STICA DE SERIES TEMPORALES HIDROL\'{O}GICAS EN LA CUENCA ALTA - MEDIA DEL RIO NEGRO, CUNDINAMARCA- META, COLOMBIA.}}
\author{Edwin Gilberto Amaya Reyes}
\date{}

\maketitle

\section*{Entregado a}
Phd. Leonardo David Donado Garz\'{o}n \\
Docente M\'{e}todos estoc\'{a}sticos en recurso hidr\'{a}ulico 

\section*{PLANTEAMIENTO DEL PROBLEMA}
La presente caracterizaci\'{o}n hidrol\'{o}gica de series de tiempo busca establecer la confiabilidad de las estaciones pluviom\'{e}tricas y pluviogr\'{a}ficos suministradas por Instituto de Hidrolog\'{i}a, Meteorolog\'{i}a y Estudios Ambientales de Colombia (IDEAM) relacionadas con la escorrent\'{i}a superficial de la cuenca media del Rio Negro con el fin de posteriormente soportar un modelo hidrol\'{o}gico de la unidad hidrogr\'{a}fica en menci\'{o}n hasta el punto sumidero o captaci\'{o}n contemplado por la informaci\'{o}n de la estaci\'{o}n Las Delicias ubicada en el municipio de Acacias Meta. Dicho soporte podr\'{a} ajustar un modelo lluvia escorrent\'{i}a en la cuenca de la quebrada Naranjal ubicada en el municipio de Quetame el cual es el objeto principal en el trabajo de profundizaci\'{o}n de maestr\'{i}a del autor del presente referido a la modelaci\'{o}n de un evento hist\'{o}rico de avenida torrencial.

La construcci\'{o}n de un modelo hidrol\'{o}gico necesita que la informaci\'{o}n incorporada se encuentre relacionada temporal y espacialmente. Por ende, el establecimiento de relaciones entre variables puede ayudar de cierta manera a estimar problemas en series incompletas dadas como por ejemplo por ausencia de datos o errores en la instrumentaci\'{o}n. La caracterizaci\'{o}n exploratoria de series hidrol\'{o}gicas temporales permite realizar un an\'{a}lisis exploratorio y estad\'{i}stico sobre los datos registrados por las estaciones con el fin de definir momentos estad\'{i}sticos, distribuciones de probabilidad y an\'{a}lisis de series que permitir\'{a}n describir patrones de tendencia, identificaci\'{o}n de eventos extremos y predicci\'{o}n del comportamiento hidrol\'{o}gico de la variable a trav\'{e}s del tiempo. 

La construcci\'{o}n del modelo hidrol\'{o}gico para el desarrollo de una simulaci\'{o}n de escenario de avenida torrencial necesita definir claramente dentro de un registro temporal de informaci\'{o}n suministrada el periodo de tiempo m\'{a}s representativo en t\'{e}rminos de disponibilidad y similitud de los registros frente al evento ocurrido hist\'{o}ricamente con el fin de calibrar los par\'{a}metros dependientes de las condiciones fisiogr\'{a}ficas y ambientales de la cuenca asociados al tr\'{a}nsito de las crecientes. Por ende, se considera muy importante realizar la caracterizaci\'{o}n de los registros de las estaciones hidrometeorol\'{o}gicas asociadas al proyecto dado que permitir\'{a}n precisar y justificar la informaci\'{o}n a implementar en los modelos hidrol\'{o}gicos de calibraci\'{o}n y simulaci\'{o}n del evento hist\'{o}rico en condici\'{o}n de aguas claras.

En la siguiente figura se muestran las unidades hidrol\'{o}gicas de modelaci\'{o}n junto con el sistema fluvial principal acompa\~{n}ado de cada una de las estaciones hidrometeorol\'{o}gicas a ser contempladas en el an\'{a}lisis de caracterizaci\'{o}n de series temporales a escala diaria suministradas por el IDEAM ,las cuales cuentan con un registro máximo comprendido entre el periodo agosto de 1951 y agosto de 2023 con interrupciones en los registros. De igual manera en la tabla inferior se aprecian los nombres de cada una de las estaciones a analizar con su correspondiente tipología y margen de análisis temporal.



\end{document}

